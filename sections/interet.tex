%%%%%%%%%%%%%%%%%%%%%%%%%%%%%%%%%%%%%%%%%%%%%%%%%%%%%%%%%%%%%%%%%%%%%%%%%%
\section{Intérêt du mode de ventilation}
%%%%%%%%%%%%%%%%%%%%%%%%%%%%%%%%%%%%%%%%%%%%%%%%%%%%%%%%%%%%%%%%%%%%%%%%%%

\begin{frame}
	\frametitle{Bénéfices escomptés}
	{\Large
	\begin{itemize}
		\item Ventilation protectrice
		\item Désencombrement
		\item Recrutement
	\end{itemize}
	}

\end{frame}

\begin{frame}
	\frametitle{Données probantes (ou pas ...)}

	\begin{block}{Études randomisées}
	\begin{table}
%	\caption{Études randomisées}
	\begin{tabular}{l l c p{5cm}}

\hline
Auteur	&	Année	&	$n$	& Clientèle\\
\hline
	Chung	&	2010	&	62	&	Grands brûlés, hôpital militaire\\
	Lucangelo	&	2009	&	44	&	Pneumonectomie (intra-op.)\\
	Bougatef	&	2007	(1989)	&	52	&	Prématurés\\
	Reper	&	2002	&	35	&	Brulure d'inhalation\\
	Platteau	&	1999	&	24	&	Chir. card. minimalement inv. (intra-op.)\\
		Hurst	&	1990	&	113	&	SDRA\\
\hline

\end{tabular}
\end{table}

	\end{block}

	\begin{block}{Séries de cas}
	\begin{table}
%	\caption{Études non randomisées}
	\begin{tabular}{l c c l}

\hline
		Auteur	&	Année	&	$n$	& Clientèle	\\
\hline
		Salim	&	2004	&	10	&	Trauma cranien en SDRA\\
		Oribabor	&	2018	&	24	& P.O. Chir. card.\\
\hline

\end{tabular}
\end{table}

	\end{block}
\end{frame}

\begin{frame}
	\frametitle{Chung et col. 2010}
	\begin{columns}[onlytextwidth]
		\column[c]{0.5\textwidth}
	\begin{block}{Caractéristiques:}

	\begin{itemize}
		\item Étude randomisée
		\item VDR-4 \textit{versus} ventilation protectrice
		\item $n=60$
		\item Pop.: brûlés avec ou sans inhalation
	\end{itemize}
	\end{block}

	\begin{block}{Résultats:}

	\begin{itemize}
		\item Mortalité et durée de ventilation inchangée
		\item Oxygénation améliorée ($p < .05$)
		\item Pression de crête et moyenne moins élevée
		\item Moins de barotrauma (0 \textit{vs} 4, $p=.04$)
		\item Moins de recours à une thérapie de secours
		\item Étude interrompue sur analyse interrimaire
	\end{itemize}
	\end{block}
		\column[c]{0.5\textwidth}
		\begin{tikzpicture}
		\begin{axis}[
				height=\textwidth,
				width=1.1\textwidth,
				xlabel=Temps (Jours),
				ylabel=$P_aO_2 / F_iO_2$,
				grid=both,
				ymin=0,
				enlarge x limits=false,
				]
			\addplot table [y=HFPV] {dat/Chung-rounded.csv};
			\addplot table [y=LTV] {dat/Chung-rounded.csv};
			\legend{HFPV, LTV};
		\end{axis}
	\end{tikzpicture}

	\end{columns}
\end{frame}

\begin{frame}
	\frametitle{Reper et col.}
	\begin{columns}[onlytextwidth]

		\column{0.5\textwidth}
	\begin{block}{Caractéristiques:}

	\begin{itemize}
		\item Étude randomisée
		\item Population: patients avec brûlure d'inhalation
		\item VDR-4 \textit{versus} ventilation conventionnelle (10 ml/kg)
		\item $n=37$
	\end{itemize}
	\end{block}

	\begin{block}{Résultats:}

	\begin{itemize}
		\item Oxygénation améliorée (p $<$ 0.05)
		\item Pressions de crète, moyenne, et expiratoire comparable
		\item Mortalitée inchangée
	\end{itemize}
	\end{block}
		\column{0.5\textwidth}
		\begin{tikzpicture}
		\begin{axis} [
				width=\textwidth,
				height=0.7\textheight,
				xlabel=Temps (Jours),
				ylabel=$P_aO_2 / F_iO_2$,
				grid=both,
				ymin=0,
				xmin=0,
				xtick={0,1,...,5},
				enlarge x limits=false,
			%	legend=true,
			%	legend image post style={mark=none}
				]
			\addplot table [y=HFPV, x expr=\thisrow{x}/24] {dat/Reper-rounded.csv};
			\addplot table [y=CMV, x expr=\thisrow{x}/24] {dat/Reper-rounded.csv};
			\legend{HFPV, CMV};
		\end{axis}
	\end{tikzpicture}
	%\footcite{Reper2002}

	\end{columns}
\end{frame}
